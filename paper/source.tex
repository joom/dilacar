\documentclass[10pt,twocolumn]{article}
\usepackage[
  margin=1.5cm,
  includefoot,
  footskip=30pt,
]{geometry}

\usepackage{layout, amsmath, ntheorem, amssymb, wasysym, stmaryrd, textgreek, bussproofs, pifont, mdframed, xcolor, hyperref}
\hypersetup{colorlinks=true,citecolor=blue}
\makeatletter
\def\mathcolor#1#{\@mathcolor{#1}}
\def\@mathcolor#1#2#3{%
  \protect\leavevmode
  \begingroup
    \color#1{#2}#3%
  \endgroup
}
\usepackage{enumitem}
\setlist[1]{itemsep=-5pt}

\usepackage[compact]{titlesec}
\titlespacing{\section}{0pt}{*0}{*0}
\titlespacing{\subsection}{0pt}{*0}{*0}
\titlespacing{\subsubsection}{0pt}{*0}{*0}

\usepackage[square,sort&compress,numbers]{natbib}
\usepackage{fontspec}
\usepackage{polyglossia}
\usepackage{bidi}
\newfontfamily\ottoman[Script=Arabic]{Scheherazade}

\setdefaultlanguage{english}

\tolerance=1
\emergencystretch=\maxdimen
\hyphenpenalty=10000
\hbadness=10000

\setmainfont{TeX Gyre Pagella}[
  Numbers=OldStyle,
]


% define a macro \Autoref to allow multiple references to be passed to \autoref
\newcommand\Autoref[1]{\@first@ref#1,@}
\def\@throw@dot#1.#2@{#1}% discard everything after the dot
\def\@set@refname#1{%    % set \@refname to autoefname+s using \getrefbykeydefault
    \edef\@tmp{\getrefbykeydefault{#1}{anchor}{}}%
    \xdef\@tmp{\expandafter\@throw@dot\@tmp.@}%
    \ltx@IfUndefined{\@tmp autorefnameplural}%
         {\def\@refname{\@nameuse{\@tmp autorefname}s}}%
         {\def\@refname{\@nameuse{\@tmp autorefnameplural}}}%
}
\def\@first@ref#1,#2{%
  \ifx#2@\autoref{#1}\let\@nextref\@gobble% only one ref, revert to normal \autoref
  \else%
    \@set@refname{#1}%  set \@refname to autoref name
    \@refname~\ref{#1}% add autoefname and first reference
    \let\@nextref\@next@ref% push processing to \@next@ref
  \fi%
  \@nextref#2%
}
\def\@next@ref#1,#2{%
   \ifx#2@ and~\ref{#1}\let\@nextref\@gobble% at end: print and+\ref and stop
   \else, \ref{#1}% print  ,+\ref and continue
   \fi%
   \@nextref#2%
}

\makeatother

% \input{glyphtounicode}
%   \pdfgentounicode=1
\usepackage{accsupp}

\theorembodyfont{\upshape}
\theoremseparator{.}
\newtheorem{lem}{Lemma}
\newtheorem{theorem}{Theorem}
\newtheorem*{corollary}{Corollary}
\newcommand{\lemautorefname}{Lemma}
\newcommand{\corollaryautorefname}{Corollary}
\theoremheaderfont{\sc}\theorembodyfont{\upshape}
\theoremstyle{nonumberplain}
\theoremseparator{:}
% \theoremsymbol{\ensuremath{\Box}}
\newtheorem{proof}{Proof}

\newcommand{\set}[1]{\{#1\}}
\newcommand{\bra}[1]{\langle#1\rangle}
\newcommand{\sep}{\;|\;}
\newcommand{\otto}[1]{\RLE{\ottoman{}\Large{}#1}}
\newcommand{\word}[1]{``\emph{#1}''}

\linespread{1.1}

% \usepackage{graphicx}
% \graphicspath{ {./images/} }

\title{Lokum: machine translation from\\Ottoman Turkish to Modern Turkish}
\author{
  Joomy Korkut\\
  \texttt{joomy@cs.princeton.edu}
}
\date{}
\begin{document}
\setlength{\abovedisplayskip}{-17pt}
\setlength{\belowdisplayskip}{0pt}
\setlength{\abovedisplayshortskip}{0pt}
\setlength{\belowdisplayshortskip}{0pt}

\maketitle

\section*{Abstract}
We present a rule-based system to translate Ottoman Turkish to Modern Turkish.
This system depends on morphological parsing of Ottoman words with the help of
dictionaries to verify the roots, and it generates the Modern Turkish
translation from the translated root by conjugating and declining the parsed
morphology into its modern spelling and pronunciation.

We use different kinds of dictionaries, one for loanwords from Arabic and
Persian, and one for Modern Turkish words. The loanword dictionary is a mapping
from the old script to the new script, while the modern dictionary is only a
set of words. In order to search a word the modern dictionary and work around
the ambiguous Ottoman spelling, we generate a pattern from the old spelling and
see what words fit that pattern.

\section{Introduction}

Ottoman Turkish is a variant of the Turkish language that was used in the
Ottoman Empire. This term usually refers to the prestige dialect used by the
educated upper class. This dialect heavily used loanwords from Arabic and
Persian, going as far as borrowing grammatical structures.~\cite{strauss2011linguistic}
Today Ottoman Turkish can also refer to the script used to write the
aforementioned prestige dialect and also the simple Turkish that was used by
others. Both were written in Perso-Arabic script, with a few adjustments to fit
the Turkish language.

When we talk about machine translation from Ottoman Turkish to Modern Turkish,
we mean a transcription system that follows the modern pronunciation of words
instead of the historical one.
For example, consider the Ottoman word \otto{ايدوب} (Eng.: ``having done").
Following the old pronunciation, this word would be transcribed \word{idüb}, while
in Modern Turkish it should be \word{edip}.

\subsection{Motivation}

Ottoman Turkish is an extinct language, but all college-level Turkish
literature students have to study it in Turkish universities.
Transcription and transliteration from Ottoman Turkish to Modern Turkish is
still a task that Turkish studies scholars and historians have to do, or hire
other people to do. It is an expensive task that takes a lot of labor-hours.
Our system aims to solve this problem by automating it.
Especially with the enhancement of OCR systems for Perso-Arabic scripts,
this is the end goal of our project.

Our system can also be a learning tool for people learning Ottoman Turkish.
Especially for non-native speakers of Turkish, the Ottoman dialect and script
is a tough nut to crack. Learners have to interact with a teacher who confirms
their transcriptions, but teachers are not available to everyone any time.
Having a system that can verify the work of learners will be quite helpful in that sense.

While scholars and historians should still learn Ottoman Turkish, there will
also be a group of users who do not know Ottoman Turkish, but still want to
understand a given text. If our system is developed enough, it will make
primary sources available to more researchers.

\section{Challenges}

Before we present the our system, it is necessary to explain the challenges of reading and translating Ottoman Turkish.

\subsection{Orthographical ambiguity}

Ottoman Turkish used Perso-Arabic script with a few extra letters and
diacritics, we will call this the Ottoman script for brevity. Modern Turkish
script, which is the Latin alphabet with extra letters, does not correspond to
the Ottoman script in a straightforward way.

There are letters in Ottoman script that correspond to many different letters in the new script.
For example, \otto{و} (named \word{vav}) corresponds to \emph{v}, \emph{o}, \emph{ö}, \emph{u} or \emph{ü} depending on the context.
Another example is \otto{ك}, which can correspond to \emph{k}, \emph{g}, \emph{ğ}, \emph{y} or \emph{n} depending on the context.
These are the most extreme cases, and while there are other such examples, we
will not go through all of them here.  Given that our end goal is to generate
Modern Turkish text, this is the kind of ambiguity we should worry about.

Similarly, there are letters in the new script that can be written with multiple letters in the Ottoman script. For example, the letter \emph{s} should be written with \otto{س}, \otto{ص} or \otto{ث}, depending on the word. Since we are not generating any text in Ottoman, we are not concerned about this kind of ambiguity.

\subsection{Missing vowels}

Many of the vowels in words are omitted in writing and inferred by the reader.
For example, the Ottoman word \otto{ترلك} (Eng.: ``slipper") should be
translated to \word{terlik}.
However, the Ottoman spelling, consists of only four letter, which are the
consonants \emph{t}, \emph{r}, \emph{l} and \emph{k}.
The vowels \emph{e} and \emph{i} are inferred by the reader, with help from
their handle on the vocabulary and their experience with the language.

\subsection{Legacy spellings of loanwords}

Ottoman Turkish preserves the original spellings of words borrowed from
Arabic and Persian, regardless of how those words are pronounced in Turkish.

Consider the Ottoman loanword from Arabic \otto{جدا} (Eng: ``seriously''). In
modern Turkish this would be written \word{cidden}. However, notice that the
Ottoman spelling consists of three letters: \otto{ج}, which corresponds to
\emph{c} here with no issues, then \otto{د}, which inexplicably is doubled in
pronunciation, and then \otto{ا}, which would normally correspond to
\emph{a} at the end of the word, but instead is pronounced \emph{en}.
For someone who cannot recognize this word, it is easy to mistranslate it to
the Persian loanword \word{cüda} (Eng: ``divided"), which is spelled the same way.

The reason for the mess here is that diacritics were almost always entirely omitted in Ottoman Turkish.
If we were to write the same word with all the diacritics, we would write
\otto{جِدَّاً}, which would disambiguate the translation.

For an example of a Persian loanword, consider the word \otto{پاپوش} (Eng: ``shoe'').
In modern Turkish this would be written \word{pabuç}, but a direct
transcription would be \word{papuş}.

Later eras of Ottoman Turkish spell the same word as \otto{پابوج}, which
directly transcribes to \word{pabuc}. Notice that the \emph{c} at the end becomes
\emph{ç} in Modern Turkish. Similarly words that \emph{d}'s at the end of words
often change to \emph{t}, and \emph{b}'s at the end change to \emph{p}.
This is a common change in Turkish grammar called
``fortitive assimilation"; loanwords in Modern Turkish reflect this in the
spelling as well, but loanwords in Ottoman Turkish keep the original spellings.

\subsection{Ambiguous word boundaries}

Not every letter in the Ottoman script attach to one another.
The letters \otto{ا}, \otto{د}, \otto{ذ}, \otto{ر}, \otto{ز}, \otto{ژ} and \otto{و}
only attach to the letter before (on the right), and not the letter after (on the left).
This is already handled perfectly by digital systems. When we are inspecting a
list of characters, we do not have to worry about what letters attach to
another, we can simply inspect them one at a time.

However, in Ottoman Turkish specifically, the letter \otto{ہ} has a conditional
attachment rule. This letter can stand for \emph{h}, \emph{a} or \emph{e} in
pronunciation. It only attaches to the left if the letter \otto{ہ} stands for
\emph{h} in that context, and does not attach to the left otherwise.

Consider the Arabic loanword \otto{الهه} (Eng: ``goddess''), \word{ilahe} in Modern Turkish.
The last two letters of this word are both \otto{ہ}. The former stands for \emph{h} and hence attaches to the latter, which stands for \emph{e}.
If we were to add a suffix to this word, such as \otto{يه}, which would be read \emph{-(y)e}, for the dative case, we would see that it does not attach to the \otto{ہ}. The final word would then be \otto{الهه‌يه} (Eng:  ``to the goddess''), \word{ilaheye} in Modern Turkish.

In order to separate that suffix from the word so that it doesn't attach, we
had to use a Unicode character called ``zero width non-joiner''. This character is invisible, but it prevents the characters before and after it from attaching.
It is also common to use a space, which is visible, to achieve the same effect.
This is a problem for our system, because Ottoman Turkish spelling normally has
spaces used as word boundaries.  However, this usage means not every space is a
word boundary, which means we have to take this into account when we translate
a word. What we think is a word might not actually be the full word.


\section{Related work}

Parsing Turkish morphology has long been an area of interest, since the
agglutinative structure of the Turkish language is an challenging research question~\cite{solak1992parsing}.

...

\section{Method and algorithms}

...

\section{Experiments and results}

...

\section{Future work}

...

\section{Conclusion}

...

\bibliography{paper}
\bibliographystyle{plainnat}

\end{document}
