\documentclass[10pt,twocolumn]{article}
\usepackage[
  margin=1.5cm,
  includefoot,
  footskip=30pt,
]{geometry}

\usepackage{layout, amsmath, ntheorem, amssymb, wasysym, stmaryrd, textgreek, bussproofs, pifont, mdframed, xcolor, hyperref, multirow}
\usepackage[super]{nth}
% \hypersetup{colorlinks=true,citecolor=blue}
\hypersetup{colorlinks=true, linkcolor=blue, citecolor=blue, urlcolor=blue, anchorcolor=blue}
\makeatletter
\def\mathcolor#1#{\@mathcolor{#1}}
\def\@mathcolor#1#2#3{%
  \protect\leavevmode
  \begingroup
    \color#1{#2}#3%
  \endgroup
}
\usepackage{enumitem}
\setlist[1]{itemsep=-5pt}

\usepackage[compact]{titlesec}
\titlespacing{\section}{0pt}{*0}{*0}
\titlespacing{\subsection}{0pt}{*0}{*0}
\titlespacing{\subsubsection}{0pt}{*0}{*0}

\usepackage{fancyhdr}
\usepackage[us,12hr]{datetime}

\usepackage[square,sort&compress,numbers]{natbib}
\usepackage{fontspec}
\usepackage{polyglossia}
\usepackage{bidi}
\newfontfamily\ottoman[Script=Arabic]{Scheherazade}

\setdefaultlanguage{english}

\tolerance=1
\emergencystretch=\maxdimen
\hyphenpenalty=10000
\hbadness=10000

\setmainfont{TeX Gyre Pagella}[
  Numbers=OldStyle,
]
\setmonofont[Scale=0.9]{Iosevka}


% define a macro \Autoref to allow multiple references to be passed to \autoref
\newcommand\Autoref[1]{\@first@ref#1,@}
\def\@throw@dot#1.#2@{#1}% discard everything after the dot
\def\@set@refname#1{%    % set \@refname to autoefname+s using \getrefbykeydefault
    \edef\@tmp{\getrefbykeydefault{#1}{anchor}{}}%
    \xdef\@tmp{\expandafter\@throw@dot\@tmp.@}%
    \ltx@IfUndefined{\@tmp autorefnameplural}%
         {\def\@refname{\@nameuse{\@tmp autorefname}s}}%
         {\def\@refname{\@nameuse{\@tmp autorefnameplural}}}%
}
\def\@first@ref#1,#2{%
  \ifx#2@\autoref{#1}\let\@nextref\@gobble% only one ref, revert to normal \autoref
  \else%
    \@set@refname{#1}%  set \@refname to autoref name
    \@refname~\ref{#1}% add autoefname and first reference
    \let\@nextref\@next@ref% push processing to \@next@ref
  \fi%
  \@nextref#2%
}
\def\@next@ref#1,#2{%
   \ifx#2@ and~\ref{#1}\let\@nextref\@gobble% at end: print and+\ref and stop
   \else, \ref{#1}% print  ,+\ref and continue
   \fi%
   \@nextref#2%
}

\makeatother

% \input{glyphtounicode}
%   \pdfgentounicode=1
\usepackage{accsupp}

\theorembodyfont{\upshape}
\theoremseparator{.}
\newtheorem{lem}{Lemma}
\newtheorem{theorem}{Theorem}
\newtheorem*{corollary}{Corollary}
\newcommand{\lemautorefname}{Lemma}
\newcommand{\corollaryautorefname}{Corollary}
\theoremheaderfont{\sc}\theorembodyfont{\upshape}
\theoremstyle{nonumberplain}
\theoremseparator{:}
\newtheorem{proof}{Proof}

\newcommand{\set}[1]{\{#1\}}
\newcommand{\bra}[1]{\langle#1\rangle}
\newcommand{\sep}{\;|\;}
\newcommand{\otto}[1]{\RLE{\ottoman{}\Large{}#1}}
\newcommand{\word}[1]{``\emph{#1}''}
\newcommand{\TODO}[1]{{\color{red}{[TODO: #1]}}}

\linespread{1.1}

% \usepackage{graphicx}
% \graphicspath{ {./images/} }

\title{Morphology and lexicon-based machine translation\\of Ottoman Turkish to Modern Turkish\footnote{Draft compiled on \today\ at \currenttime.}}
\author{
  Joomy Korkut\\
  \normalsize Princeton University\\
  \normalsize \texttt{joomy@cs.princeton.edu}
}
\date{}
\begin{document}
\setlength{\abovedisplayskip}{-17pt}
\setlength{\belowdisplayskip}{0pt}
\setlength{\abovedisplayshortskip}{0pt}
\setlength{\belowdisplayshortskip}{0pt}

\maketitle

\section*{Abstract}
We present a rule-based system to translate Ottoman Turkish to Modern Turkish.
This system depends on partial morphological parsing of Ottoman words with the help of
dictionaries to verify the roots, and it generates the Modern Turkish
translation from the translated root by conjugating and declining the parsed
morphology into its modern spelling and pronunciation.

We use two dictionaries, one for loanwords from Arabic and
Persian, and one for Modern Turkish words. Searching in the loanword dictionary
is straightforward, but for the modern dictionary we derive a pattern from the
old spelling and search using that pattern.

\section{Introduction}

Ottoman Turkish is a variant of the Turkish language that was used in the
Ottoman Empire. This term usually refers to the prestige dialect used by the
educated upper class. This dialect heavily used loanwords from Arabic and
Persian, going as far as borrowing grammatical
structures~\cite{redhouse1884simplified, hagopian1907ottoman, strauss2011linguistic}.
Today Ottoman Turkish can refer to the script used to write the
aforementioned prestige dialect and also the simple Turkish that was used by
others. Both were written in Perso-Arabic script, with a few adjustments to fit
the Turkish language.
However, despite being used as the default script from \nth{13} to \nth{20}
century, there were no standard spelling rules for Ottoman Turkish.

In this project, we will prioritize the spelling that was common
in the late \nth{19} and early \nth{20} century.

When we talk about machine translation from Ottoman Turkish to Modern Turkish,
we mean a transcription system that follows the modern pronunciation of words
instead of the historical one.
For example, consider the Ottoman word \otto{ايدوب} (Eng.: ``having done").
Following the old pronunciation, this word would be transcribed \word{idüb}, while
in Modern Turkish it should be \word{edip}.

In this paper, we present a rule-based algorithm to achieve translation from
Ottoman Turkish to Modern Turkish.  We will show that partial morphological
parsing of Ottoman Turkish is easier because of its syntax rules that do not
reflect pronunciation. Ottoman Turkish suffixes are often not conjugated or
declined, which makes it easier for a computer to parse.
Once we strip a word of its suffixes, we search for the root in our
dictionaries, one for loanwords from Arabic and Persian, and one for Modern
Turkish words. The loanword dictionary is a mapping
from the old script to the new script, while the modern dictionary is only a
set of words. In order to search a word the modern dictionary, we generate a
pattern from the old spelling and see what words fit that pattern.
The pattern is based on what sounds that an Ottoman letter can correspond to in
that context, and where it can take extra vowels.

\subsection{Motivation}

Ottoman Turkish is an extinct language, but all college-level Turkish
literature and history students have to study it in Turkish universities.
Transcription and transliteration from Ottoman Turkish to Modern Turkish is
still a task that Turkish studies scholars and historians have to do, or hire
other people to do. It is an expensive task that takes a lot of labor-hours.
Our system aims to solve this problem by automating it.
We envision that our system will be used alongside optical character
recognition (OCR) systems for Perso-Arabic script, which can take a scanned image of
text and turn it into a digital format.
Then our system can take the digital Ottoman text and translate into Modern Turkish.

Our system can also be a learning tool for people studying Ottoman Turkish.
Especially for non-native speakers of Turkish, the Ottoman dialect and script
is a tough nut to crack. Students have to interact with a teacher who confirms
their transcriptions, but teachers are not available to everyone all the time.
Their other option is to use an already existing transcription, which also might not be available.
Having a system that can verify the work of learners will be quite helpful in that sense.

While scholars and historians should still learn Ottoman Turkish, there will
also be a group of users who do not know Ottoman Turkish, but still want to
understand a given text. If our system is developed enough, it will make
primary sources available to more researchers.\footnote{Given that they are
available in digital format. OCR systems only work well on printed
(\emph{matbu}) material; other writing styles and handwriting require a
different kind of expertise.}

We preferred a rule based approach due to the lack of data in Ottoman
Turkish. The amount of digitized texts alongside their translation or
transcription is very limited in the first place.
Resources in Ottoman Turkish are often either in image form, or
untranslated.\footnote{The OpenOttoman project (\url{https://openottoman.org/})
and Ottoman WikiSource (\url{https://wikisource.org/wiki/Osmanlıca_VikiKaynak})
are good examples of both.}
Furthermore, we would need enough data to account for all the grammatical
irregularities. Unfortunately, this is not possible in the current state of
Ottoman text digitization.

\section{Challenges}

Before we present our system, it is necessary to explain the challenges of reading and translating Ottoman Turkish.

\subsection{Orthographical ambiguity}

Ottoman Turkish used Perso-Arabic script with a few extra letters and
diacritics, we will call this the Ottoman script for brevity. Modern Turkish
script, which is the Latin alphabet with extra letters, does not correspond to
the Ottoman script in a straightforward way.

There are letters in Ottoman script that correspond to many different letters in the new script.
For example, \otto{و} corresponds to \emph{v}, \emph{o}, \emph{ö}, \emph{u} or \emph{ü} depending on the context.
Another example is \otto{ك}, which can correspond to \emph{k}, \emph{g}, \emph{ğ}, \emph{y} or \emph{n} depending on the context.
These are the most extreme cases, and while there are other such examples, we
will not go through all of them here.  Given that our end goal is to generate
Modern Turkish text, this is the kind of ambiguity we should worry about.

Similarly, some letters in the new script can be written with multiple letters
in the Ottoman script. For example, the letter \emph{s} should be written with
\otto{س}, \otto{ص} or \otto{ث}, depending on the word. Since we are not
generating any Ottoman Turkish text, we are not concerned about this kind of
ambiguity.

\subsection{Missing vowels}
\label{missing-vowels}

Many of the vowels in words are omitted in writing and inferred by the reader.
For example, the Ottoman word \otto{ترلك} (Eng.: ``slipper") should be
translated to \word{terlik}.
However, the Ottoman spelling consists of only four letters, which are the
consonants \emph{t}, \emph{r}, \emph{l} and \emph{k}.
The vowels \emph{e} and \emph{i} are inferred by the reader, with help from
their handle on the vocabulary and their experience with the language.

\subsection{Legacy spellings of loanwords}

Ottoman Turkish preserves the original spellings of words borrowed from
Arabic and Persian, regardless of how those words are pronounced in Turkish.

Consider the Ottoman loanword from Arabic \otto{جدا} (Eng: ``seriously''). In
modern Turkish this would be written \word{cidden}. However, notice that the
Ottoman spelling consists of three letters: \otto{ج}, which corresponds to
\emph{c} here with no issues, then \otto{د}, which inexplicably is doubled in
pronunciation, and then \otto{ا}, which would normally correspond to
\emph{a} at the end of the word, but instead is pronounced \emph{en}.
For someone who cannot recognize this word, it is easy to miss the translation above and go for
the Persian loanword \word{cüda} (Eng: ``separated"), which is spelled the same way.

The reason for the mess here is that diacritics are almost always entirely omitted in Ottoman Turkish.
If we were to write the same word with all the diacritics, we would write
\otto{جِدًّا}, which would disambiguate the translation.

For an example of a Persian loanword, consider \otto{پاپوش} (Eng: ``shoe'').
A direct transcription would be \word{papuş},
but in Modern Turkish this would be written \word{pabuç}.

Later eras of Ottoman Turkish spell the same word as \otto{پابوج}, which
directly transcribes to \word{pabuc}. Notice that the \emph{c} at the end becomes
\emph{ç} in Modern Turkish. Similarly \emph{d}'s at the end of words
often change to \emph{t}, and \emph{b}'s at the end change to
\emph{p}.\footnote{Such last letters can change back if if the word takes a
suffix that starts with a vowel. The word \emph{pabuç} with the dative suffix
\emph{-a} would be \word{pabuca}. This is an example of ``consonant lenition''.}
This is a common change in Turkish grammar called
``fortitive assimilation"; loanwords in Modern Turkish reflect this in the
spelling as well, but loanwords in Ottoman Turkish keep the original spellings.

\subsection{Ambiguous word boundaries}

The Ottoman script is written from right to left, by attaching letters to each other.
However, not every letter attach to one another.
% Not every letter in the Ottoman script attach to one another.
The letters \otto{ا}, \otto{د}, \otto{ذ}, \otto{ر}, \otto{ز}, \otto{ژ} and \otto{و}
only attach to the letter before (on the right), and not the letter after (on the left).
This is already handled perfectly by digital systems. When we are inspecting a
list of characters, we do not have to worry about what letters attach to
another, we can simply inspect them one at a time.

However, in Ottoman Turkish specifically, the letter \otto{ہ} has a conditional
attachment rule. This letter can stand for \emph{h}, \emph{a} or \emph{e} in
pronunciation. It only attaches to the left if the letter \otto{ہ} is read
\emph{h} in that context, and does not attach to the left otherwise.

Consider the loanword from Arabic \otto{الهه} (Eng: ``goddess''), \word{ilahe} in Modern Turkish.
The last two letters of this word are both \otto{ہ}. The former is in its
middle form \otto{ـهـ}, since it stands for \emph{h}. It attaches to the
latter, is in its final form \otto{ـه} since it stands for \emph{e}.
If we were to add a suffix to this word, such as \otto{يه}, which would be read \emph{-(y)e}, for the dative case, we would see that it does not attach to the \otto{ہ}. The final word would then be \otto{الهه‌يه} (Eng:  ``to the goddess''), \word{ilaheye} in Modern Turkish.

In order to separate that suffix from the word so that it doesn't attach, we
had to use a Unicode character called ``zero width non-joiner''. This character
is invisible, but it prevents the characters before and after it from
attaching.
It is also common to use a space, which is visible, to achieve the same effect.
This is a problem for our system, because Ottoman Turkish spelling normally has
spaces as word boundaries.  However, this usage means not every space is a
word boundary, which means we have to take this into account when we translate
a word. What we think is a word might not actually be the full word.

\section{Related work}

\subsection{Morphological parsing}

Parsing Turkish morphology has long been an area of interest in computational
linguistics, since the agglutinative structure of the Turkish language
is an challenging research question. Therefore there is a long line of work in this area~\cite{hankamer1986finite, solak1992parsing, oflazer1994two, oflazer1994outline, hakkani2002statistical, eryigit2004affix, sak2007morphological, sak2009stochastic, coltekin2010freely}, among which it is possible to find both finite-state-based approaches and statistical approaches.
Grammar of the Turkish language is unusually regular, which might have
encouraged researchers to insist on finite-state methods.

Our approach in this paper is inspired from \citet{eryigit2004affix}, which
describes a finite-state machine to parse words in reverse, by stripping
affixes and reaching a root. Their approach does not use a lexicon, which is
different from our approach.

\subsection{Machine translation of Ottoman Turkish}

Our project is not the first to attempt machine translation from Ottoman
Turkish to Modern Turkish. We have found at least two such projects.

One of them was developed by Rahmi Tura, whose web site is not
reachable anymore. He had a demo video online, which showcased features like
translation from Ottoman Turkish to Modern Turkish with or without the
diacritics and vice versa. However, the program was never made available or
sold online. In our personal communication\footnote{On April 29th, 2019, via e-mail.}, he stated that
his system has a database that consists of approximately one million words, in
their conjugated and declined forms. He argued that matching words in their
whole form increases the accuracy, even though it is at the expense of
developing a comprehensive database.

The other such project is called Dervaze, developed by \citet{dervaze}.
Their project is very similar to ours, and they have a working demo system
online.\footnote{\url{http://dervaze.com/translate-ott/}}
We have tested their implementation, and we encountered some problems that we claim stem for their pipeline design:
\begin{itemize}[noitemsep,topsep=0pt]
\item They choose to pick only one translation out of many possible ones. Our project will give the full range of possibilities to the user, which reduces readability of the result but increases accuracy.
\item They skip words that fail to translate. We will always report to the user if we fail on a word.
\item They fail on translating simple Turkish words. This is because they use a Ottoman to Turkish dictionary which might not include simpler Turkish words.
\end{itemize}

\section{Method and algorithms}

\subsection{Overview}

Despite the challenges mentioned above, our algorithm is based on a simple
principle: Turkish is an agglutinative language where roots of words are not
inflected when new suffixes are attached, and the new suffixes take forms
according to the last syllable of the word that are attached to. We will take a
word and perform a partial morphological parsing by break it apart to its roots
and suffixes, then we will find the translation of the root and reconstruct the
suffixes according to the grammar rules of Modern Turkish.

Take the plural suffix for nouns, written as \emph{-ler} or \emph{-lar} in Modern
Turkish. If we attach this suffix to the word \word{göz} (Eng: ``eye''),
then we get \word{gözler}, because the last syllable\footnote{Syllabification
rules of Turkish dictate that there is only one vowel per syllable.} before the
suffix contains one of the front vowels \emph{e}, \emph{i}, \emph{ö}, \emph{ü}.
But if we attach the same suffix to the word \word{kuş} (Eng: ``bird''),
then we get \word{kuşlar}, because the previous syllable has one of the back
vowels \emph{a}, \emph{ı}, \emph{ö}, \emph{ü}.

The plural suffix in Ottoman Turkish, however, does not change its form. It is
written as \otto{-لر} for all cases, only containing the letters for the
\emph{l} and \emph{r}.
The word \word{gözler} would be written as \otto{گوزلر} and \word{kuşlar} would
be written as \otto{قوشلر}.
The missing vowels problem we described in \autoref{missing-vowels} work in our
advantage here; the computer can just look for the \otto{-لر} at the end, strip it off, and
use the rest as an hypothetical root.

An hypothetical root would be looked up in the dictionaries, and if there is
not entry for that, we would continue to look for more suffixes. Even if there
is an entry, it is possible that a word has multiple translations in Modern
Turkish, so we should still look for more suffixes.

There are a few corner cases in the \otto{-لر} (\emph{-ler}, \emph{-lar})
example. One of them is that the ending with this letter sequence does not
necessarily mean it is a plural suffix. It is possible to have words that end
with the same letters that have no suffixes. Consider the loanword from French
\otto{پوپولر} (Eng: ``popular''), which translates to Modern Turkish
\word{popüler}. We can try to parse the ending as a suffix and search for the
beginning in the dictionary, which will fail. However, our program can find a
word in one our dictionaries that matches either directly or through a pattern,
which lets the program recover from the previous state.

% will have an entry
% for \otto{پوپولر} in one of our dictionaries,

The other corner case is that \otto{-لر} (\emph{-ler}, \emph{-lar}) is not just
the plural suffix for nouns. The \nth{3} person plural suffix in verb
conjugations is also of the same form. For example, the word \otto{گلدى} (Eng: ``he/she/it came'', Tur: \word{geldi}) can take the same suffix and become \otto{گلديلر} (Eng: ``they came'', Tur: \word{geldiler}).
Here the fact that our morphological parsing is partial saves us.
When we are parsing the suffixes our goal is not to produce a full
morphological structure like \emph{göz<n><pl>} for \word{gözler} or
\emph{gel<v><past><3p>} for \word{geldiler}. Our goal is merely to identify what
suffixes we have so that we can reconstruct them, we are not interested in what
function they have in the grammar.

Most suffixes in Ottoman Turkish generally have only one form, even though their Modern Turkish equivalents might have many more.\footnote{This is because the Modern Turkish spelling closely reflects the pronunciation.} For example, the suffix of the direct past tense \otto{-دى} in Ottoman Turkish has 8 different forms in Modern Turkish:
\emph{-dı}, \emph{-di}, \emph{-du}, \emph{-dü}, \emph{-tı}, \emph{-ti}, \emph{-tu} and \emph{-tü}.
Notwithstanding, there are exceptions to these cases. Rarely it is possible to
see \otto{-دى} as \otto{-دو}, \otto{-تى} or \otto{-تو}, which correspond to the
forms in Modern Turkish. This is due to the lack of standardization in Ottoman
Turkish; a writer's spelling depends on the era they live in and also their
``idiography", i.e.\ an individual's unique way of spelling.

\subsection{Dictionary lookup and pattern generation}

As we described earlier, our program will use two kinds of dictionaries to look
up words. The first one is a dictionary that maps Ottoman Turkish spellings to
Modern Turkish spellings, it is basically a table with two columns.
It will have a mapping from \otto{الهه} to \word{ilâhe}, from \otto{جدا} to
\word{cidden} and so on. The second dictionary is a table with only one column,
it is simply a list of words in Modern Turkish.

The first dictionary mostly consists of loanwords from Arabic and Persian, since
they are spelled in Ottoman Turkish the same way they are spelled in the
original language, even when their pronunciation is changed in Turkish.
Therefore, unless there is a dictionary the program can look up from, it will
get those word wrong.
Consider the word \otto{چهارشنبه} (Eng: ``wednesday''). It is spelled
\word{çarşamba} Modern Turkish, but a direct transcription would be
\word{çeharşenbe}.

For word that are originally Turkish, there is a much higher chance that we can
get the right translation from the spelling, despite the ambiguities.
We will generate a regular expression from the Ottoman spelling and match it
against the entries in our second dictionary.

% Consider the word \otto{آرابه} (Eng: ``car'' or ``cart''). In Modern Turkish
% this is translated to \word{araba}.
Consider the word \otto{گوز} (Eng: ``eye'', Tur: \word{göz}) that we mentioned above.
There are three letters in this word: % \otto{گ}, \otto{و} and \otto{ز}.
\begin{enumerate}[noitemsep,topsep=0pt]
  \item \otto{گ} only corresponds to one letter in Modern Turkish, which is
    \emph{g}. Hence our pattern for this letter only consists of \texttt{/g/}.
  \item \otto{و} can correspond to five letters in Modern Turkish, \emph{v},
    \emph{o}, \emph{ö}, \emph{u} or \emph{ü}. The last four of these are
    vowels, and if \otto{و} here stands for a vowel, then it does not imply the
    possible existence of another vowel before or after it. However, if
    \otto{و} stands for \emph{v} here, it may or may not be followed or
    preceded by a vowel, and we have to account for that vowel in our pattern.

    Hence our pattern for this letter would be
    \texttt{/(((a|e|i|ı|o|ö|u|ü)?v(a|e|i|ı|o|ö|u|ü)?)|o|ö|u|ü)/}.
    We have optional vowel patterns before and after the \emph{v}, or this
    letter might just be standing for \emph{o}, \emph{ö}, \emph{u} or \emph{ü}
\item \otto{ز} only corresponds to one letter in Modern Turkish, which is
    \emph{z}. Hence our pattern for this letter only consists of \texttt{/z/}.
\end{enumerate}

Therefore, our final pattern for this word will be
\texttt{/\^{}g(((a|e|i|ı|o|ö|u|ü)?v(a|e|i|ı|o|ö|u|ü)?)|o|ö|u|ü)z\$/}.

Searching with this pattern in the modern dictionary, we will find two matching
words: \word{göz} and \word{güz}. This ambiguity is inherent to the Ottoman
script, and there is not much we can do about that. Our program will take this
root and decline the suffixes if there are any, and display both translations
to the end user.

\subsection{Walking through examples}

\TODO{gotta write}

\section{Experiments and results}

\TODO{gotta write}

\section{Future work}

\TODO{gotta write}

\section{Conclusion}

\TODO{gotta write}

\section*{Acknowledgments}

We would like to thank Nilüfer Hatemi and Duygu Coşkuntuna
from the Near Eastern Studies department at Princeton
for their comments on the draft of this paper.

\bibliography{paper}
\bibliographystyle{plainnat}

% \onecolumn
\clearpage
\appendix
\section{Appendix}
Below you can find the Ottoman Turkish script, with each letter in its
isolated, final, medial and initial forms, along with the letter names and the
letter(s) they correspond to in Modern Turkish~\cite{bugday2014introduction}.

\begin{center}
\begin{tabular}{|c|c|c|c|c|c|}
\hline
\textbf{Iso} & \textbf{Fin} & \textbf{Med} & \textbf{Init} & \textbf{Name}  & \textbf{Tur} \\ \hline
\otto{ا}  & \otto{ـا}  & \multicolumn{2}{c|}{--}            & elif           & a, e                    \\ \hline
% \otto{ء}  & \multicolumn{3}{c|}{--}                         & hemze          & --, '                   \\ \hline
\otto{ب}  & \otto{ـب}  & \otto{ـبـ}            & \otto{بـ}  & be             & b (p)                   \\ \hline
\otto{پ}  & \otto{ـپ}  & \otto{ـپـ}            & \otto{پـ}  & pe             & p                       \\ \hline
\otto{ت}  & \otto{ـت}  & \otto{ـتـ}            & \otto{تـ}  & te             & t                       \\ \hline
\otto{ث}  & \otto{ـث}  & \otto{ـثـ}            & \otto{ثـ}  & se             & s                       \\ \hline
\otto{ج}  & \otto{ـج}  & \otto{ـجـ}            & \otto{جـ}  & cim            & c                       \\ \hline
\otto{چ}  & \otto{ـچ}  & \otto{ـچـ}            & \otto{چـ}  & çim            & ç                       \\ \hline
\otto{ح}  & \otto{ـح}  & \otto{ـحـ}            & \otto{حـ}  & ha             & h                       \\ \hline
\otto{خ}  & \otto{ـخ}  & \otto{ـخـ}            & \otto{خـ}  & hı             & h                       \\ \hline
\otto{د}  & \otto{ـد}  & \multicolumn{2}{c|}{--}            & dal            & d                       \\ \hline
\otto{ذ}  & \otto{ـذ}  & \multicolumn{2}{c|}{--}            & zel            & z                       \\ \hline
\otto{ر}  & \otto{ـر}  & \multicolumn{2}{c|}{--}            & re             & r                       \\ \hline
\otto{ز}  & \otto{ـز}  & \multicolumn{2}{c|}{--}            & ze             & z                       \\ \hline
\otto{ژ}  & \otto{ـژ}  & \multicolumn{2}{c|}{--}            & je             & j                       \\ \hline
\otto{س}  & \otto{ـس}  & \otto{ـسـ}            & \otto{سـ}  & sin            & s                       \\ \hline
\otto{ش}  & \otto{ـش}  & \otto{ـشـ}            & \otto{شـ}  & şın            & ş                       \\ \hline
\otto{ص}  & \otto{ـص}  & \otto{ـصـ}            & \otto{صـ}  & sad            & s                       \\ \hline
\otto{ض}  & \otto{ـض}  & \otto{ـضـ}            & \otto{ضـ}  & dad            & d, z                    \\ \hline
\otto{ط}  & \otto{ـط}  & \otto{ـطـ}            & \otto{طـ}  & tı             & t                       \\ \hline
\otto{ظ}  & \otto{ـظ}  & \otto{ـظـ}            & \otto{ظـ}  & zı             & z                       \\ \hline
\otto{ع}  & \otto{ـع}  & \otto{ـعـ}            & \otto{عـ}  & ayn            & ', --                   \\ \hline
\otto{غ}  & \otto{ـغ}  & \otto{ـغـ}            & \otto{غـ}  & gayn           & g, ğ, v                 \\ \hline
\otto{ق}  & \otto{ـق}  & \otto{ـفـ}            & \otto{فـ}  & fe             & f                       \\ \hline
\otto{ق}  & \otto{ـق}  & \otto{ـقـ}            & \otto{قـ}  & kaf            & k                       \\ \hline
\otto{ك}  & \otto{ـك}  & \otto{ـكـ}            & \otto{كـ}  & kef            & k                       \\ \hline
\otto{گ}  & \otto{ـگ}  & \otto{ـگـ}            & \otto{گـ}  & gef            & g, ğ                    \\ \hline
\otto{ڭ}  & \otto{ـڭ}  & \otto{ـڭـ}            & \otto{ڭـ}  & nef            & n                       \\ \hline
\otto{ل}  & \otto{ـل}  & \otto{ـلـ}            & \otto{لـ}  & lam            & l                       \\ \hline
\otto{م}  & \otto{ـم}  & \otto{ـمـ}            & \otto{مـ}  & mim            & m                       \\ \hline
\otto{ن}  & \otto{ـن}  & \otto{ـنـ}            & \otto{نـ}  & nun            & n                       \\ \hline
\otto{و}  & \otto{ـو}  & \multicolumn{2}{c|}{--}            & vav            & v, o, ö, u, ü           \\ \hline
\otto{ه}  & \otto{ـه}  & \otto{ـهـ}            & \otto{هـ}  & he             & h, e, a                 \\ \hline
\otto{ی}  & \otto{ـی}  & \otto{ـیـ}            & \otto{یـ}  & ye             & y, i, ı                 \\ \hline
\end{tabular}
\end{center}

The letters \otto{گ} (gef) and \otto{ڭ} (nef) are often written as \otto{ك}
(kef), since they are derived from it. The reader has to infer which one is
meant. Also \otto{ه} (he) does not have medial or initial forms if it stands for \emph{e} or \emph{a}.

% Please add the following required packages to your document preamble:
% \usepackage{multirow}

\newpage
\vspace*{0.51em}

In order to disambiguate the pronunciation, there are some available diacritics as well:
\begin{center}
\begin{tabular}{|c|c|c|}
\hline
\textbf{Sign} & \textbf{Name} & \textbf{Function} \\ \hline
\otto{ء} & hemze & indicates an initial vowel \\ \hline
\otto{ـَ} & üstün & a, e \\ \hline
\otto{ـِ} & esre & ı, i \\ \hline
\otto{ـُ} & ötre & o, ö, u, ü \\ \hline
\otto{ـّ} & şedde & doubled consonant \\ \hline
\otto{ـْ} & sükun & absence of vowel \\ \hline
\otto{ـٓ} & medde & long a \\ \hline
\otto{ـً‎} & \multirow{3}{*}{tenvin} & -an, -en \\ \cline{1-1} \cline{3-3}
\otto{ـٍ‎} & & -in \\ \cline{1-1} \cline{3-3}
\otto{ـٌ‎} & & -un, -ün \\ \hline
\end{tabular}
\end{center}

The ones other than \otto{ء} (hemze) and \otto{ـٓ} (medde) are usually omitted.



\end{document}
